\documentclass[a4paper]{article}
\usepackage[slovene]{babel}
\usepackage[utf8]{inputenc}
\usepackage[T1]{fontenc}
\usepackage[pdftex]{graphicx}
\usepackage{amsmath}
\usepackage{titlesec}
\usepackage{enumerate}
\usepackage{mathpazo} 
\usepackage{blindtext}

\title{Predpisana drevesa z najmanjšim/največjim Wienerjevim indeksom}
\author{Klemen Hovnik\\ Matija Gubanec Hančič \\ Jan Rudof}

\begin{document}
\maketitle

\section{Uvod}

Naj bo $G=(V(G),  E(G))$ enostaven povezan neusmerjen graf. $Wienerjev$ $ indeks$ (oziroma $Wienerjevo$  $"stevilo$ $ W(G)$) je definiran kot
\begin{equation}
W(G) = \frac{1}{2}\sum_{u\in V(g)}\sum_{v\in V(g)} d_G(u,v).
\end{equation}
Tukaj označimo z $d_G(u,v)$ razdaljo med vozliščem $u$ in $v$ v grafu $G$. 
\\
\\
Naša naloga je, da analiziramo lastnosti dreves z določenim številom vozlišč in 
fiksno maksimalno stopnjo vozlišč, ki imajo najmanjši Wienerjev indeks. Podobno nas zanimajo tudi lastnosti 
dreves na določenem številu vozlišč s fiksnim premerom, ki imajo največji možni Wienerjev indeks.

\section{Opis dela}

Za izvedbo projekta smo si izbrali programski jezik $Sage$, saj ta že vsebuje orodja za delo z grafi, prav tako 
pa ima tudi generator dreves in že vgrajeno funkcijo za izračun Wienerjevega indeksa. Najprej smo se lotili 
izračuna Wienerjevih indeksov na preprostih grafih z malo vozlišči, da vidimo, kako naj bi ta struktura grafov 
z minimalnimi oziroma maksimalnimi indeksi izgledala v splošnem. 
\\
\\
Za iskanje najmanjših Wienerjevih indeksov pri določenem številu vozlišč in pri fiksni maksimalni stopnji 
smo naredili slovar slovarjev, kamor smo shranjevali optimalno dobljena drevesa. Podobno smo naredili 
slovar slovarjev za drevesa z maksimalnim Wienerjevim indeksom pri določenem številu vozlišč in pri 
fiksnem premeru grafa. Seveda smo to naredili le za grafe z malo vozlišči, saj se kaj kmalu izkaže, da je 
naš algoritem za izračun indeksov časovno prepotraten.
\\
\\
Tako smo si pripravili vse potrebno za analizo strukture dreves, ki nam bo pomagala kasneje pri iskanju 
indeksov večjih grafov. Ideja je, da bi glede na dobljene rezultate na grafih z manj vozlišči lahko predvidevali,
kako naj bi ta drevesa izgledala v splošnem.

\section{Načrt za nadaljnje delo}

Naslednji korak našega dela bo konstrukcija $genetskega$ $algoritma$.
\\
\\
$Genetski$ $algoritem$ je metahevristika, navdihnjena s strani procesov naravne selekcije in spada v razred $razvojnih$
$algoritmov$. Uporablja se za generiranje kvalitetnih rešitev v optimizaciji, ki temeljijo na operatorjih kot so mutacija, križanje
in selekcija.  
\\
\\
V genetskem algoritmu se uporabi množica kandidatov za rešitev, ki jih nato razvijamo do optimalne rešitve. Vsak kandidat
ima določene lastnosti, katere lahko spremenimo oziroma lahko mutirajo. Evolucija rešitev se ponavadi začne na naključni izbiri kandidatov,
katere potem s pomočjo iteracije razvijamo. Na vsakem iterativnem koraku se potem oceni primernost novih kandidatov za optimizacijski
problem. Najboljše kandidate potem uporabimo za naslednji korak iteracije in tako dalje. Na koncu se algoritem zaključi,
ko doseže maksimalno število iteracijskih korakov oziroma, ko dobi najboljši približek optimalni rešitvi.
\\
\\
Naša začetna množica kandidatov bodo optimalna drevesa, ki smo jih dobili z našim prvim enostavnim algoritmom. 
Nato bomo ustvarili genetski algoritem, ki bo iz teh začetnih podatkov generiral nova optimalna drevesa. Ta postopek bomo
nadaljevali (in pri tem selekcionirali iz novo nastalih dreves le najboljše za naslednje korake), dokler ne bomo prišli do optimalnih 
dreves za določeno število vozlišč. Pri tem bomo morali paziti, da se bo ohranjala maksimalna stopnja vozlišč, 
oziroma v drugem primeru, premer. 
\\
\\
Na koncu bomo primerjali algoritma in pogledali, kdaj se ustavi naš enostavni algoritem za iskanje grafov z max/min Wienerjevim indeksom
oziroma za katero število vozlišč je genetski algoritem še učinkovit.





\end{document}
