\documentclass[a4paper]{article}
\usepackage[slovene]{babel}
\usepackage[utf8]{inputenc}
\usepackage[T1]{fontenc}
\usepackage[pdftex]{graphicx}
\usepackage{amsmath}
\usepackage{titlesec}
\usepackage{enumerate}
\usepackage{mathpazo} 
\usepackage{blindtext}

\title{Predpisana drevesa z najmanjšim/največjim Wiener indeksom}
\author{Klemen Hovnik\\ Matija Gubanc Hančič \\ Jan Rudof}

\begin{document}
\maketitle

\section{Uvod}

Naj bo $G=(V(G),  E(G))$ enostaven povezan neusmerjen graf. $Wienerjev$ $ indeks$, oziroma $Wienerjevo$  $"stevilo$ $ W(G)$, je definiran kot
\begin{equation}
W(G) = \frac{1}{2}\sum_{u\in V(g)}\sum_{v\in V(g)} d_G(u,v).
\end{equation}
Tukaj označimo z $d_G(u,v)$ razdaljo med vozliščem $u$ in $v$ v grafu $G$. \\
\\
Naša naloga je, da analiziramo lastnosti dreves z določenim številom vozlišč in 
fiksno maksimalno stopnjo vozlišč, ki imajo najmanjši Wienerjev indeks. Podobno nas zanimajo tudi lastnosti dreves na določenem številu vozlišč s \\
fiksnim premerom, ki imajo največji možni Wienerjev indeks.

\section{Opis dela}

Za izvedbo projekta smo si izbrali programski jezik $Sage$, saj ta že vsebuje orodja za delo z grafi, prav tako pa ima tudi generator dreves 
in že vgrajeno funkcijo za izračun Wienerjevega indeksa. Najprej smo se lotili izračuna Wienerjevih indeksov na preprostih grafih z malo vozlišči, 
da vidimo kako naj bi ta struktura izgledala v splošnem.

\section{Načrt za nadaljnje delo}

Ideja je, da bi glede na dobljene rezultate na grafih z manj vozlišči, lahko predvidevali kako naj bi ta drevesa izgledala v splošnem.
Poskusili bomo z genetskimi algoritmi križati drevesa, za katera menimo, da bi lahko dosegla najmanjši oziroma največji možni
Wienerjev indeks.




\end{document}