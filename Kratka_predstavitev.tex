\documentclass[a4paper]{article}
\usepackage[slovene]{babel}
\usepackage[utf8]{inputenc}
\usepackage[T1]{fontenc}
\usepackage[pdftex]{graphicx}
\usepackage{amsmath}
\usepackage{titlesec}
\usepackage{enumerate}
\usepackage{mathpazo} 
\usepackage{blindtext}

\title{Predpisana drevesa z najmanjšim/največjim Wienerjevim indeksom}
\author{Klemen Hovnik\\ Matija Gubanc Hančič \\ Jan Rudof}

\begin{document}
\maketitle

\section{Uvod}

Naj bo $G=(V(G),  E(G))$ enostaven povezan neusmerjen graf. $Wienerjev$ $ indeks$ (oziroma $Wienerjevo$  $"stevilo$ $ W(G)$) je definiran kot
\begin{equation}
W(G) = \frac{1}{2}\sum_{u\in V(g)}\sum_{v\in V(g)} d_G(u,v).
\end{equation}
Tukaj označimo z $d_G(u,v)$ razdaljo med vozliščem $u$ in $v$ v grafu $G$. 
\\
\\
Naša naloga je, da analiziramo lastnosti dreves z določenim številom vozlišč in 
fiksno maksimalno stopnjo vozlišč, ki imajo najmanjši Wienerjev indeks. Podobno nas zanimajo tudi lastnosti 
dreves na določenem številu vozlišč s fiksnim premerom, ki imajo največji možni Wienerjev indeks.

\section{Opis dela}

Za izvedbo projekta smo si izbrali programski jezik $Sage$, saj ta že vsebuje orodja za delo z grafi, prav tako 
pa ima tudi generator dreves in že vgrajeno funkcijo za izračun Wienerjevega indeksa. Najprej smo se lotili 
izračuna Wienerjevih indeksov na preprostih grafih z malo vozlišči, da vidimo, kako naj bi ta struktura grafov 
z minimalnimi oziroma maksimalnimi indeksi izgledala v splošnem. 
\\
\\
Za iskanje najmanjših Wienerjevih indeksov pri določenem številu vozlišč in pri fiksni maksimalni stopnji 
smo naredili slovar slovarjev, kamor smo shranjevali optimalno dobljena drevesa. Podobno smo naredili 
slovar slovarjev za drevesa z maksimalnim Wienerjevim indeksom pri določenem številu vozlišč in pri 
fiksnem premeru grafa. Seveda smo to naredili le za grafe z malo vozlišči, saj se kaj kmalu izkaže, da je 
naš algoritem za izračun indeksov časovno prepotraten.
\\
\\
Tako smo si pripravili vse potrebno za analizo strukture dreves, ki nam bo pomagala kasneje pri iskanju 
indeksov večjih grafov. Ideja je, da bi glede na dobljene rezultate na grafih z manj vozlišči lahko predvidevali,
kako naj bi ta drevesa izgledala v splošnem.

\section{Načrt za nadaljnje delo}

Glede nadaljnjega dela smo prišli do dveh idej, kako bi lahko iskali optimalne Wienerjeve indekse pri večjem številu vozlišč:
\\
\\
Prva ideja je, da bi s pomočjo $dinami"cnega$ $programiranja$ in že znanih grafov, ki imajo optimalne indekse,
iskali optimalne Wienerjeve indekse še za grafe z večjim številom vozlišč. Že znane grafe bi tako gradili do grafov 
z željenim številom vozlišč in željeno maksimalno stopnjo oziroma premerom. Gradili bi jih tako, da bi že znanim 
grafom dodajali povezave do željenege števila vozlišč, pri tem pa pazili, da ohranjamo željeno maksimalno stopnjo 
oziroma premer. Seveda bi nas pri gradnji novih grafov zanimali le tisti z minimalnim oziroma maksimalnim Wienerjevim indeksom.
\\
\\
Druga ideja je, da bi si pomagali z $genetskim$ $algoritmom$.  Za začetno populacijo dreves bi vzeli optimalna drevesa,
ki smo jih predhodno našli z našim zamudnim algoritmom, potem pa bi na vsakem nadaljnjem koraku iz teh dreves 
generirali novo populacijo dreves. Pri teh bomo izločili najslabša, ostala pa bomo poskusili križati in generirati nova drevesa.
Pri tem načinu se bo število vozlišč ohranilo, paziti pa bomo morali, da se bo ohranjala tudi maksimalna stopnja oziroma premer.





\end{document}
